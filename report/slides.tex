\documentclass{beamer}

\usepackage[utf8]{inputenc}
\usepackage[french]{babel}

\usetheme{Warsaw}

\title{Projet microprocesseur}
\author{Mihai Dusmanu, Clément Pascutto}
\institute{Système digital\\École normale supérieure}
\date{Mardi 26 Janvier 2016}

\begin{document}

	\begin{frame}
		\maketitle
	\end{frame}

	\begin{frame}
		\frametitle{Sommaire}
		\tableofcontents
	\end{frame}

	\section{Microprocesseur}
	\subsection{Assembleur}
	\begin{frame}
		\frametitle{Assembleur}
		\framesubtitle{Jeu d'instructions de base :
			\emph{brainfuck}(1993)}
		\begin{enumerate}
			\item $+$ : Incrémente le compteur courant 
			\pause
			\item $-$ : Décrémente le compteur courant 
			\pause
			\item $>$ : Incrémente le pointeur vers le compteur courant
			\pause
			\item $<$ : Décrémente le pointeur vers le compteur courant
			\pause
			\item $[$ : Saute à l'instruction suivante si la valeur du compteur
				courant est non nulle, sinon saute au $]$ correspondant
			\pause
			\item $]$ : Saute au $[$ correspondant
			\pause
			\item $.$ : Affiche la valeur du compteur courant
			\pause
			\item $,$ : Lit une valeur et l'écrit dans le compteur courant
		\end{enumerate}
	\end{frame}

	\begin{frame}
		\frametitle{Assembleur}
		\framesubtitle{Optimisations}
		On apporte deux optimisations majeures :
		\pause
		\begin{itemize}
			\item Contraction : $+++++++$ devient $7+$.
				\\
				On peut appliquer la même transformation aux opérations $+$, 
				$-$, $>$ et $<$.
			\pause
		\item Ajout d'un registre : $[->+<]$ devient \#$->$\$$+<$. \\
			\# vaut la valeur du compteur courant. \\
			\$ vaut la valeur du dernier \# calculé.
		\end{itemize}
	\end{frame}

	\subsection{Architecture}
	\begin{frame}
		\frametitle{Architecture}
        \begin{itemize}
            \item Une mémoire ROM contenant les instructions du programme
            \item Une table de routage composée d'une ROM qui convertit les instructions en bits de contrôle pour les autres composants
            \item Une unité arithmétique (UA) capable d'effectuer les opérations PLUS et MOINS
            \item Une mémoire RAM contenant $2^{16}$ compteurs
            \item Un pointeur indiquant l'adresse du compteur courant dans la RAM
            \item Un pointeur indiquant l'adresse de l'instruction courante dans la ROM du programme relié à un incrémenteur conditionnel
            \item Une RAM de taille 1 contenant la valeur du dernier \# calculé
            \item Un entrée sur 16 bits, prise en compte seulement lorsque l'instruction courante est la lecture
            \item Une sortie sur 16 bits, qui vaut $2^{16}-1$ si l'instruction courante n'est pas une instruction PRINT, et qui vaut la valeur du compteur courant sinon
        \end{itemize}
	\end{frame}

	\subsection{Compilation de l'assembleur}
	\begin{frame}
		\frametitle{Compilation de l'assembleur}
		Les instructions de la ROM sont des mots de $20$ bits :
		\pause
		\begin{itemize}
			\item $3$ bits d'instructions
			\pause
			\item $1$ bit de registre
			\pause
			\item $16$ bits de paramètres
		\end{itemize}
		\pause
		~\\
		On compile \#$op$ par \#$+$\$$op$ pour mettre à jour la valeur du
		registre.\pause \\
		On compile $[$ par $0+[$ pour calculer le flag Zero.
	\end{frame}
\end{document}
