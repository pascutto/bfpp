\documentclass{beamer}

\usepackage[utf8]{inputenc}
\usepackage[french]{babel}

\usetheme{Warsaw}

\title{Projet microprocesseur}
\author{Mihai Dusmanu, Clément Pascutto}
\institute{Système digital\\École normale supérieure}
\date{Mardi 26 Janvier 2016}

\begin{document}

	\begin{frame}
		\maketitle
	\end{frame}

	\begin{frame}
		\frametitle{Sommaire}
		\tableofcontents
	\end{frame}

	\section{Microprocesseur}
	\subsection{Assembleur}
	\begin{frame}
		\frametitle{Assembleur}
		\framesubtitle{Jeu d'instructions de base :
			\emph{brainfuck}(1993)}
		\begin{enumerate}
			\item $+$ : Incrémente le compteur courant 
			\pause
			\item $-$ : Décrémente le compteur courant 
			\pause
			\item $>$ : Incrémente le pointeur vers le compteur courant
			\pause
			\item $<$ : Décrémente le pointeur vers le compteur courant
			\pause
			\item $[$ : Saute à l'instruction suivante si la valeur du compteur
				courant est non nulle, sinon saute au $]$ correspondant
			\pause
			\item $]$ : Saute au $[$ correspondant
			\pause
			\item $.$ : Affiche la valeur du compteur courant
			\pause
			\item $,$ : Lit une valeur et l'écrit dans le compteur courant
		\end{enumerate}
	\end{frame}

	\begin{frame}
		\frametitle{Assembleur}
		\framesubtitle{Optimisations}
		On apporte deux optimisations majeures :
		\pause
		\begin{itemize}
			\item Contraction : $+++++++$ devient $7+$.
				\\
				On peut appliquer la même transformation aux opérations $+$, 
				$-$, $>$ et $<$.
			\pause
		\item Ajout d'un registre : $[->+<]$ devient \#$->$\$$+<$. \\
			\# vaut la valeur du compteur courant. \\
			\$ vaut la valeur du dernier \# calculé.
		\end{itemize}
	\end{frame}

	\subsection{Architecture}
	\begin{frame}
		\frametitle{Architecture}
        \begin{itemize}
            \item Une ROM de programme
				\pause
            \item Une ROM de routage
				\pause
			\item Une unité arithmétique
				\pause
            \item Un pointeur indiquant l'adresse de l'instruction
				\pause
            \item Une mémoire contenant la valeur du dernier \# calculé
				\pause
			\item Un entrée sur 16 bits
            \item Une sortie sur 16 bits
				\pause
			\item La RAM principale contenant $2^{16}$ compteurs
            \item Un pointeur indiquant l'adresse du compteur courant dans la RAM
        \end{itemize}
	\end{frame}

	\subsection{Compilation de l'assembleur}
	\begin{frame}
		\frametitle{Compilation de l'assembleur}
		Les instructions de la ROM sont des mots de $20$ bits :
		\pause
		\begin{itemize}
			\item $3$ bits d'instructions
			\pause
			\item $1$ bit de registre
			\pause
			\item $16$ bits de paramètres
		\end{itemize}
		\pause
		~\\
		On compile \#$op$ par \#$+$\$$op$ pour mettre à jour la valeur du
		registre.\pause \\
		On compile $[$ par $0+[$ pour calculer le flag Zero.
	\end{frame}

	\section{Montre digitale}
	\subsection{Présentation générale}
	\begin{frame}
		\frametitle{Présentation générale}
		La mémoire contient les compteurs suivants :
		\pause
		\begin{enumerate}
			\setcounter{enumi}{-1}
			\item $60 -$ secondes
				\pause
			\item $60 -$ minutes
			\item $24 -$ heures
			\item nombre de jours dans le mois $-$ jours
			\item $12 -$ mois
			\item année
			\item bissextile ?
		\end{enumerate}
	\end{frame}

	\subsection{Sleep}
	\begin{frame}
		\frametitle{Sleep}
		
		\begin{itemize}
			\item On lit l'entrée
				\pause
			\item On la compare à la valeur attendue
				\pause
			\item Si elles correspondent, on continue et on change la
				valeur attendue
				\pause
			\item Sinon, on recommence
		\end{itemize}
		\pause
		$[,$\\
			\hspace{1cm} \#$->>$\$$+<$\#$-$\$$+>$\$$-[<<+>>$\#$-]<<$\\ 
			// Test d'inégalité\\
		$]$\\
		$>-$\#$->$\$$-$\#$-<$\$$+<$ // Mise à jour de la valeur attendue
	\end{frame}

	\subsection{Années bissextiles}
	\begin{frame}
		\frametitle{Années bissextiles}
		On ajoute dans la mémoire :
		\pause
		\begin{enumerate}
			\setcounter{enumi}{6}
			\item $4 -$ année $\mod 4$
				\pause
			\item $100 -$ année $\mod 100$ 
				\pause
			\item $400 -$ année $\mod 400$
				\pause
		\end{enumerate}
		\vspace{5mm}
		Le code pour changer l'année devient :
			$+>>->->-$
	\end{frame}
	
	\begin{frame}
		\frametitle{Années bissextiles}
		année bissextile $\Leftrightarrow (v(7)==0) + (v(8) ==0) +
		(v(9)==0) \in \{1,3\}$\\
		\pause
		\vspace{1cm}
		Pour traiter les mois :
		\pause
		\begin{itemize}
			\item On place dans la mémoire à un offset $o$ les longueurs
				des mois :\pause \\
				$o + 2k$ : longueur du mois $k$ si l'année n'est pas bissextile\\
				\pause $o + 2k + 1$
				\pause
			\item On se déplace de $o + 2\times mois + bissextile$
				
		\end{itemize}
	\end{frame}
\end{document}
