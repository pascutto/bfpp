\documentclass{beamer}

\usepackage[utf8]{inputenc}
\usepackage[french]{babel}

\usetheme{Warsaw}

\title{Projet microprocesseur}
\author{Mihai Dusmanu, Clément Pascutto}
\institute{Système digital\\École normale supérieure}
\date{Mardi 26 Janvier 2016}

\begin{document}

	\begin{frame}
		\maketitle
	\end{frame}

	\begin{frame}
		\frametitle{Sommaire}
		\tableofcontents
	\end{frame}

	\section{Microprocesseur}
	\subsection{Assembleur}
	\begin{frame}
		\frametitle{Assembleur}
		\framesubtitle{Jeu d'instructions de base :
			\emph{brainfuck}(1993)}
		\begin{enumerate}
			\item $+$ : Incrémente le compteur courant 
			\pause
			\item $-$ : Décrémente le compteur courant 
			\pause
			\item $>$ : Incrémente le pointeur vers le compteur courant
			\pause
			\item $<$ : Décrémente le pointeur vers le compteur courant
			\pause
			\item $[$ : Saute à l'instruction suivante si la valeur du compteur
				courant est non nulle, sinon saute au $]$ correspondant
			\pause
			\item $]$ : Saute au $[$ correspondant
			\pause
			\item $.$ : Affiche la valeur du compteur courant
			\pause
			\item $,$ : Lit une valeur et l'écrit dans le compteur courant
		\end{enumerate}
	\end{frame}

	\begin{frame}
		\frametitle{Assembleur}
		\framesubtitle{Optimisations}
		On apporte deux optimisations majeures :
		\pause
		\begin{itemize}
			\item Contraction : $+++++++$ devient $7+$.
				\\
				On peut appliquer la même transformation aux opérations $+$, 
				$-$, $>$ et $<$.
			\pause
		\item Ajout d'un registre : $[->+<]$ devient \#$->$\$$+<$. \\
			\# vaut la valeur du compteur courant. \\
			\$ vaut la valeur du dernier \# calculé.
		\end{itemize}
	\end{frame}

	\subsection{Architecture}
	\begin{frame}
		\frametitle{Architecture}

	\end{frame}

	\subsection{Compilation de l'assembleur}
	\begin{frame}
		\frametitle{Compilation de l'assembleur}
		Les instructions de la ROM sont des mots de $20$ bits :
		\pause
		\begin{itemize}
			\item $3$ bits d'instructions
			\pause
			\item $1$ bit de registre
			\pause
			\item $16$ bits de paramètres
		\end{itemize}
		\pause
		~\\
		On compile \#$op$ par \#$+$\$$op$ pour mettre à jour la valeur du
		registre.\pause \\
		On compile $[$ par $0+[$ pour calculer le flag Zero.
	\end{frame}
\end{document}
